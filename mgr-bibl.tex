% -- latex --
% Literatura (obowi�zkowo):
%
\begin{thebibliography}{99}
% Pozycja bibliograficzna typu: ksi��ka.
% autor/autorzy: tytu�, wydawca, rok-wydania
\bibitem{tei.intro}
Burnard L.,
Sperberg-McQueen C. M.:
\textit{TEI Lite: An Introduction to Text Encoding for Interchange},
1995.
\url{http://www.uic.edu/orgs/tei/intros/teiu5.html}.

% Czasami zamiast autora-osoby nale�y umie�ci� nazw� organizacji/firmy,
% tak jak w przyk�adzie poni�ej
\bibitem{pdf.spec}
Adobe Systems Inc:
\textit{Portable Document Format Reference Manual Version 1.2},
Addison-Wesley, 1996.

% Pozycja bibliograficzna typu: artyku�.
% autor/autorzy: ,,tytu�-artyku�u'', tytu�-czasopisma numer, rok, strony
\bibitem{beebe}
Beebe N.~H.~F.:
,,Bibliography Prettyprinting and Syntax Checking'',
\textit{TUGBoat} 14/4, 1995, 
s.~395--419.

% Pozycja bibliograficzna typu: rozdzia� z pracy zbiorowej/monografii.
% autor/autorzy: ,,tytu�-rozdzia�u'', w: tytu�-ksi��ki
%   (red. redaktor), miejsce-wydania rok-wydania, strony
\bibitem{braams}
Braams J. L.:
The Status of Babel,
w:~\textit{Proceedings of the IXth European TeX Conference}
(W. Dol red.), Arnhem 1995, s.~17--26.

\bibitem{XMLspec}
Bray T.,
Paoli J.,
Sperberg-McQueen C.~M.:
\textit{Extensible Markup Language (XML)~1.0},
1998. \url{http://www.w3.org/TR/1998/REC-xml-19980210}.

\bibitem{bryan}
Bryan M.:
\textit{SGML and HTML explained},
Addison-Wesley
1997.

\bibitem{xslt}
Clark J.:
\textit{XSL Transformations (XSLT) Version~1.0},
1999. \url{http://www.w3.org/TR/xslt}.

\bibitem{p.flynn}
Flynn P.:
\textit{Understanding SGML and XML tools},
Kluwer Academic Publishers
1998.

\bibitem{Goldfarb}
Goldfarb C.:
\textit{The SGML Handbook},
Oxford University Press.

\bibitem{lwc}
Goossens M.,
Rahtz S.:
\textit{The {\LaTeX} Web Companion},
Addison-Wesley
1999.

\bibitem{python.tkinter}
Grayson J.:
\textit{Python and Tkinter programming},
Manning Publications.

\bibitem{Herwinen-Pract}
Van Herwijnen E.:
\textit{Practical SGML},
Kluwer Academic Publishers
1990.

\bibitem{lamport}
Lamport L.:
\textit{{\LaTeX}: a~document preparation system},
Addison-Wesley
1994.

\bibitem{maler.ex}
Maler E.:
\textit{SGML exceptions and XML},
1998.
\url{http://www.arbortext.com/Think_Tank/XML_Resources/SGML_Exceptions_and_XML}.

\bibitem{maler.devel}
Maler E.,
El Andaloussi J.:
\textit{Developing SGML DTDs. From Text to Model to Markup},
Prentice Hall
1996.

\bibitem{s.north}
North S.:
\textit{XML dla ka�dego},
Helion
2000.

\bibitem{whirlwind-guide}
Pepper S.:
\textit{Whirlwind Guide to SGML Tools and Vendors}.
\url{http://www.infotek.no/sgmltool/index.htm}.

\bibitem{p.perl}
Wall L.,
Christiansen T.,
Schwartz R.:
\textit{Programming Perl},
O'Reilly \&~Associates
1996.

\bibitem{db.guide}
Walsh N.,
Muellner L.:
\textit{DocBook: The Definitive Guide},
O'Reilly \&~Associates October~1999.
Dokument dost�pny tak�e w~\url{http://www.docbook.org/}.
\end{thebibliography}
